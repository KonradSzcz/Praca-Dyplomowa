\clearpage
\chapter{Wprowadzenie}
\section{Definicja kryptowalut}
Kryptowaluty to cyfrowe aktywa, które wykorzystują kryptografię do zabezpieczenia transakcji oraz kontrolowania nowych jednostek waluty. Jedną z największych cech kryptowalut jest ich decentalizacja, co oznacza brak centalnego organu kontrolującego, który nadzorowałby ich emisję i obrót. To jest jeden z charakterystycznych elementów odróżniających kryptowaluty od tradycyjnych walut emitowanych przez państwa. 

Kryptowaluty są oparte na technologii blockchain co jest jedną z ich kluczowych cech. Technologia ta stanowi rozproszoną i niemal niezmienialną bazę danych, zapewniającą bezpieczeństwo i chroniącą integralność transakcji. Każda transakcja kryptowalutowa jest zapisywana w blokach, które następnie są powiązane ze sobą łańcuchowo, tworząc niezmienną historię transakcji.

Istnieje wiele różnych kryptowalut, które często wprowadzają unikalne cechy technologiczne, przystosowane do rozmaitych zastosowań i potrzeb rynku. Przykłady kryptowalut, które cieszą się dużą popularnością poza wszyskim dobrze znanym Bitcoinem to Etherum, które jest znane z tworzenia i wykonania inteligentnych kontaktów, które automatyzują procesy w różnych dziediznach oraz Ripple, który cechuje się ułatwianiem płatności międzynarodowych.

\section{Rola kryptowalut w dzisiejszym świecie}
W dzisiejszym pężnie rozwijającym się cyfrowym świecie kryptowaluty stanowią nieodłączną część przekształceń zarówno w sferze finansów, jak i technologii. Od chwili wprowadzenia Bitcoina przez grupę lub osobę Satoshi Nakamoto w 2009, kryptowaluty rozszerzyły zakres możliwości inwestycyjnych i technologicznych dodatkowo przyczyniając się do zmian w tradycyjnych systemach płatności. 

Jednym z kluczowych aspektów roli kryptowalut jest ich funkcja jako alternatywnych form płatności. Kryptowaluty pozwalają na szybkie, bezpieczne i bezpośrednie przesyłanie wartości, pomijając tradycujne pośrednictwo instytucji finansowych. Jest to szczególnie ważny aspekt, ponieważ w erze cyfryzacji coraz częściej odbywają się transakcje międzynarodowe, które wymagają skutecznych i efektywnych rozwiązań płatniczych.

Kolejnym ważnym aspektem są otwierające się, dzięki kryptowalutom, nowe horyzonty inwestycyjne dla osób poszukujących alternatywnych źródeł lokowania kapitału. Ciągłe wahania cenowe oraz wysokie stopy zwrotu przykuwają uwagę zarówno doświadczonych inwestorów, jak i osób stawiających swoje pierwsze kroki na ryneu finansowym. Dodatkowo, kryptowaluty pozwalają na dywersyfikację portfela inwestycyjnego, co prowadzi do zmniejszenia ryzyka na rynku jak i do zwiększenia bezpieczeństwa przed tradycyjnymi zagrożeniami rynkowymi.

Trzecim równie istotnym aspektem roli kryptowalut jest ich działanie na rozwój technologii blockchain. Technologia ta, będąca fundamentem większości kryptowalut, rozpowszechnia się poza granice sektora finansowego i znajduje zastosowanie w innych dziedzinach takich jak zdrowie czy logistyka. Dzięki rozwojowi kryptowalut technologia blockchain cały czas staje się coraz bardziej powszechna i jest coraz częściej brana pod uwagę jako potencjalne narzędzie do poprawy efektywności różnych procesów biznesowych.

Dzięki wciąż rosnącej popularności cyfrowych aktywów nie tylko zmieniają się tradycyjne podejścia do inwestycji i płatności, ale również widoczny jest rozwój technologiczny. Są poniekąd katalizatorem zmian w dzisiejszym globalnym technologicznym oraz finansowym.

\section{Zapotrzebowanie na edukację o kryptowalutach}
W obliczu dyczmicznego rozwoju w dzeidzinie finansów i technologi, edukacja w zakresie kryptowalut nie tylko jest korzystna ale wręcz staje się niezbędna. Coraz więcej osób pragnie zrozumieć mechanizmy działania kryptowalut oraz poznać potencjalne ryzyko i korzyści z nimi związane, a także zastanawia się nad sposobem praktycznego wykorzystania tych cyfrowych aktywów.

Wzrost zainteresowania kryptowalutami wynika z ich rosnącej popularności jako alternatywy dla tradycyjnych systemów płatności i inwestycji. Wraz z dynamicznym rozwojem rynku cyfrowych aktywów, wzrasta potrzeba zwiększenia wiedzy na temat sposobów ich funkcjonowania oraz potencjalnych scenariuszy wykorzystania. Coraz więcej osób zdaje sobie sprawę, że edukacja w zakresie kryptowalut jest niezbędna dla podejmowania przemyślanych i odpowiedzialnych decyzji inwestycyjnych oraz dla zrozumienia potencjalnych przemian, jakie te technologie mogą spowodować w przyszłości. 

Wzrost zainteresowania rynkiem kryptowalut wiąże się również ze wzrostem liczby oszustw i manipulacji, zrozumenie ryzyka związaniego związanego z tym specyficznym rynkiem staje się niezbędne dla zapewnieniea bezpieczeństwa finansowego. Osoby pragnące aktywnie inwestować w kryptowaluty i korzystać z nich coraz bardziej potrzebują solidnej wiedzy, która umożliwi im podejmowanie odpowiedzialnych dezycji oraz zabezpieczenie się przed ewentualnymi zagrożeniami. 

W rezultacie rośnie zapotrzebowanie na różnorodne, łatwo dostępne formy edukacji w tym temacie. Dostarczenie rzetelnych, przystępnych i interaktywnych materiałów edukacyjnych może pomóc w bzwiększaniu świadomości i umiejętności potrzebnych do efektywnego korzystania z kryptowalut w dzisiejszym świecie cyfrowym. 

\section{Opis wykorzystanych technologii}

Android Studio \parencite{www:androidstudio} - jest to zintegrowane środowisko programistyczne przeznaczone specjalnie do twoorzenia aplikacji na platformę Android. Oferuję dużą gamę różnych narzędziu do projekktowania interfejsu użytkownika, debugowania, testowania oraz kompilowania aplikacji. Android Studio jest darmowym oprogramowaniem, które dzięki oficjalnemu wsparciu ze strony Google jest jednym z najbardziej rozpoznawalnych środowisk dla programistów Android.

Kotlin \parencite{www:kotlin} - jest to nowoczesny język programowania opracowany przez firmę JetBrains. Jest czytelniejszą i łatwiejszą do zrozumienia alternatywą języka Java na platformę Android. Jego główną zaletą jest bogata składnia, która usprawnia proces programowania ponadto oferuje wsparcei dla programowania obiektowego, funkcjonalnego oraz reaktywnego. 

SQLite \parencite{www:sqlite} - jest lekką oraz często wykorzystywaną w aplikacjach mobilnych relacyjną bazą danych. Jej głównymi zaletami jest łatwość wdrożenia, niewielkie wymagania zasobów oraz obsługa standardowego języka zapytań SQL. Pomimo braku skomplikowania SQlite oferuje wiele rozbudowanych funkcji, przykładem tego są transakcje ACID, które zapewniają spójność danych  i bezpieczeństwo operacji. 